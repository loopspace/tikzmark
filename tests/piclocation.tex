\documentclass{article}

\usepackage{tikz}
\usetikzlibrary{tikzmark}

\makeatletter
\tikzset{
  pic anchor/.initial={(0,0)},
  adjust pic position/.code={
    \tikz@scan@one@point\pgfutil@firstofone(pic cs:\tikz@fig@name-origin)\relax
    \pgf@xa=\pgf@x
    \pgf@ya=\pgf@y
    \tikz@scan@one@point\pgfutil@firstofone(pic cs:\tikz@fig@name-anchor)\relax
    \advance\pgf@xa by -\pgf@x
    \advance\pgf@ya by -\pgf@y
    \tikzset{
      shift={(\the\pgf@xa,\the\pgf@ya)},
      execute at end scope={%
        \tikzmark{\tikz@fig@name-origin}{(0,0)}%
        \tikzmark{\tikz@fig@name-anchor}{\pgfkeysvalueof{/tikz/pic anchor}}%
      }
    }
  },
  location/.pic={
    \begin{scope}[adjust pic position]
    \draw (0,0) -- (2,3);
    \node (-A) at (2,1) {A};
    \end{scope}
  }
}
\makeatother
\begin{document}

\begin{tikzpicture}
\draw
(0,0) circle[radius=6pt]
(2,1) circle[radius=6pt]
;
\pic[name=test,pic anchor={(-A)}] at (2,1) {location};
\draw[red] (pic cs:test-origin) -- (pic cs:test-anchor);
\begin{scope}[shift={(2,3)},rotate=30,scale=2]
\fill
(1,0) circle[radius=3pt]
;
\draw (pic cs:where is it) -- ++(0,1);
\tikzmark{where is it}{(1,0)}
\end{scope}
\draw (pic cs:where is it) -- ++(0,-1);
\end{tikzpicture}
\end{document}
